%!TEX program = xelatex
\documentclass[12pt, pdf]{style} 
%Wähle oben entweder pdf oder print, default ist print. Pdf generiert einseite Dokumente, die an Bildschirmen besser aussehen, print ist für zweiseitigen druck (Buchdruck) und lässt daher Kapitel immer auf der rechten Seite beginnen. Das kann in PDFs nervige Leerseiten erzeugen, daher kann man das hier einstellen.
\tikzset{external/prefix=tikz/Main/}
\begin{otherlanguage}{ngerman}
\iffalse ANMERKUNGEN

- die Vorlage muss folgendermaßen kompiliert werden da sonst unter Umständen die Abkürzungen oder das Literaturverzeichnis nicht funktionieren:
    1. XeLaTex
    2. Makeglossaries
    3. Biber
    4. XeLaTex
    5. XeLaTex
- die Formatierung kann in style.cls editiert werden
- Referenzen im BibTeX Format können z. B. über Google Scholar gefunden und in libraryJR.bib eingefügt werden
\fi
\end{otherlanguage}

\graphicspath{{4_images/}}
\addbibresource{3_backmatter/library/libraryJR.bib}

\newcommand{\FirstName}{Vorname}
\newcommand{\LastName}{Nachname}
\newcommand{\Matriculation}{Matrikelnummer}
\newcommand{\Place}{Ort}
\newcommand{\Signature}{\centering\includegraphics[height=1.5 cm]{Unterschrift.png}}
\newcommand{\Date}{\today}
\newcommand{\titleEN}{Titel auf Englisch}
\newcommand{\titleDE}{Titel auf Deutsch}
\newcommand{\instituteEN}{Institutname auf Englsich}
\newcommand{\instituteDE}{Institutname auf Deutsch}
\newcommand{\texttypeDE}{Art der Arbeit auf Deutsch}
\newcommand{\texttypeEN}{Art der Arbeit auf Englisch}
\newcommand{\supervisorA}{Name}
\newcommand{\supervisorB}{Name}
\newcommand{\profA}{Name}
\newcommand{\profB}{Name}


\begin{document}
% Titlepage, table of contents, etc.
    \frontmatter
    \pagenumbering{Roman}  
    \begin{titlepage}
        \begin{center}
{\Large RWTH Aachen University}\\[2cm]
{\Huge Titel der Arbeit}\\[1cm]
{\large Bachelorarbeit / Masterarbeit}\\[2cm]
Autor: Dein Name\\
Betreuer: Dr. Mustermann\\
Datum: \today
\end{center}
    \end{titlepage}
    \include{1_frontmatter/2_Abstract}
    %!TEX root = ../../main.tex

% Zweite Seite des Dokuments mit Aufgabenstellung

\begin{center}
\huge{Task Description}


\end{center}
\normalsize 


\cleardoublepage

    
    %!TEX root = ../../main.tex

% Seite mit eidesstattlicher Erklärung
\begin{otherlanguage}{ngerman}

% Erkärung nach Vorlage von AVT/PT
\begin{center} {\LARGE Eidesstattliche Erklärung} \end{center}


\noindent\rule[1ex]{30ex}{1pt} \hspace{15ex} \rule[1ex]{30ex}{1pt}\vspace{-0.2cm}\\
\phantom{}\hspace{6ex} Name, Vorname  \hspace{20.5ex} Matrikelnummer (freiwillige Angabe)\vspace{0.1cm}\\
Ich versichere hiermit an Eides Statt, dass ich die vorliegende Arbeit/Bachelorarbeit/Masterarbeit* mit dem Titel \vspace{0.2cm}\\
\noindent \rule[1ex]{\textwidth}{1pt}\\
\noindent \rule[1ex]{\textwidth}{1pt}\\
\noindent \rule[1ex]{\textwidth}{1pt}\\
selbstständig und ohne unzulässige fremde Hilfe (insbes. akademisches Ghostwriting) erbracht habe.
Ich habe keine anderen als die angegebenen Quellen und Hilfsmittel benutzt; dies umfasst
insbesondere auch Software und Dienste zur Sprach-, Text- und Medienproduktion. Ich erkläre, dass
für den Fall, dass die Arbeit in unterschiedlichen Formen eingereicht wird (z.B. elektronisch, gedruckt,
geplottet, auf einem Datenträger) alle eingereichten Versionen vollständig übereinstimmen. Die Arbeit
hat in gleicher oder ähnlicher Form noch keiner Prüfungsbehörde vorgelegen.\vspace{-0.2cm}\\

\noindent\rule[1ex]{30ex}{1pt} \hspace{15ex} \rule[1ex]{30ex}{1pt}\vspace{-0.2cm}\\
\phantom{}\hspace{6ex} Ort, Datum  \hspace{37.5ex} Unterschrift \vspace{0.1cm}\\
*Nichtzutreffendes bitte streichen\vspace{0.1cm}\\
\textbf{Belehrung:}\\
\textbf{\S 156 StGB: Falsche Versicherung an Eides Statt}

\noindent Wer vor einer zur Abnahme einer Versicherung an Eides Statt zuständigen Behörde eine solche Versicherung
falsch abgibt oder unter Berufung auf eine solche Versicherung falsch aussagt, wird mit Freiheitsstrafe bis zu drei
Jahren oder mit Geldstrafe bestraft.

\textbf{\S 161 StGB: Fahrlässiger Falscheid; fahrlässige falsche Versicherung an Eides Statt}

\noindent(1) Wenn eine der in den \S\S 154 bis 156 bezeichneten Handlungen aus Fahrlässigkeit begangen worden ist, so tritt Freiheitsstrafe bis zu einem Jahr oder Geldstrafe ein.

\noindent (2) Straflosigkeit tritt ein, wenn der Täter die falsche Angabe rechtzeitig berichtigt. Die Vorschriften des § 158 Abs. 2 und 3 gelten entsprechend.

\noindent Die vorstehende Belehrung habe ich zur Kenntnis genommen:\vspace{0.6cm}\\
\noindent\rule[1ex]{30ex}{1pt} \hspace{15ex} \rule[1ex]{30ex}{1pt}\vspace{-0.2cm}\\
\phantom{}\hspace{6ex} Ort, Datum  \hspace{37.5ex} Unterschrift \vspace{0.1cm}\newpage

\begin{center} {\LARGE Declaration} \end{center}

I hereby declare to have listed all aids used in this thesis alongside their application below. This concerns in particular software and services for language, text, and media production. 
I have been made aware by my supervisor of the proper and improper use of such aids in the thesis.\newline
In this thesis, I have
\begin{itemize}
	\item[$\Box$] In particular not used software and services for language, text, and media production. 
	\item[$\Box$] 
	used the following aids (Please also list how you applied each aid):\\\\


\centering
\begin{tabular}{lll} 
\toprule
Section & Tool(s) used & Reason for use  \\
\midrule
1       & DEEPl        & Grammar/ Spell checker, Reformulating            \\
            
\bottomrule
\end{tabular}
%* Please fill out the table in Excel using the Abrreviations specified on this page, then import the table in https://www.latex-tables.com/
% Press Auto-Booktabs, then press generate text.
% Lastly, please add the command \midpointrule after the first line, e.g.,
% Section & Tool Used & Reasoning  \\\midrule 
\end{itemize}


 \vspace{2cm}

%\noindent\rule[1ex]{30ex}{1pt} \hspace{15ex} \rule[1ex]{30ex}{1pt}\vspace{-0.2cm}\\
%\phantom{}\hspace{6ex} Ort, Datum  \hspace{37.5ex} Unterschrift \vspace{0.1cm}
%
%\newpage
%
%\begin{center} {\LARGE Erklärung} \end{center}
%
%Ich erkläre mich hiermit einverstanden, dass die vorliegende Arbeit in der Lehrstuhlbibliothek aufbewahrt wird und kopiert werden darf und habe keine Einwände gegen eine Veröffentlichung der gesamten Arbeit oder von einzelnen Teilen daraus.\vspace{0.5cm} \\
%
%Ich bin damit einverstanden, dass die RWTH Aachen folgende Daten zu meiner Person im Internet veröffentlicht: Name, Vorname, Titel der Arbeit. \\ 
%
%Mir ist bekannt, 
%\begin{itemize}
%	\item[$\Box$] 
%	dass ich diese Einwilligung jederzeit schriftlich mit Wirkung für die Zukunft widerrufen kann und meine elektronisch gespeicherten Daten unverzüglich gelöscht werden müssen und 
%	\item[$\Box$] 
%	die Aachener Verfahrenstechnik bei Widerruf meiner Einwilligung verpflichtet ist, mein Werk aus der Bibliothek des Instituts/Lehrstuhls zu entfernen.
%\end{itemize}
%
% \vspace{2cm}
%
%\noindent\rule[1ex]{30ex}{1pt} \hspace{15ex} \rule[1ex]{30ex}{1pt}\vspace{-0.2cm}\\
%\phantom{}\hspace{6ex} Ort, Datum  \hspace{37.5ex} Unterschrift \vspace{0.1cm}
\end{otherlanguage}
%
%\cleardoublepage
    \tableofcontents

% Thesis content - Add your chapters here
    \mainmatter
    \onehalfspacing
    
%   \include{0_templates/About_Latex}    
%  \include{0_templates/About_Scientific-writing} 
    
    \chapter{Einleitung}
Hier beginnt deine Einleitung...
    \chapter{Theoretischer Hintergrund}
Beschreibe hier die Theorie...
%    \include{2_chapter/3_Concept/Hypothesis} %optional
    \chapter{Materialien und Methoden}
Beschreibe hier die Methoden...
    \chapter{Ergebnisse und Diskussion}
Präsentiere deine Ergebnisse...
    \chapter{Fazit und Ausblick}
Ziehe hier dein Fazit...
    
% Nomenclature, bibliography, etc.
%    \tikzset{external/prefix=tikz/Acknoledgement/}
\chapter{Acknowledgement}
    %!TEX root = ./main.tex
\tikzset{external/prefix=tikz/Nomenclature/}
% Abkürzungen definieren
\newacronym{cvd}{CVD}{Chemical Vapour Deposition}
\newacronym{lpcvd}{LPCVD}{Low Pressure \acrshort{cvd}}
\newacronym{apcvd}{APCVD}{Atmospheric-Pressure \acrshort{cvd}}
\newacronym{mocvd}{MOCVD}{Metal-organic \acrshort{cvd}}
\newacronym{ald}{ALD}{Atomic Layer Deposition}
\newacronym{cvi}{CVI}{Chemical Vapour Infiltration}
\newacronym{tgcvi}{TG-CVI}{Thermal-Gradient Chemical Vapour Infiltration}
\newacronym{sem}{SEM}{Scanning Electron Microscopy}
\newacronym{fgm}{FGM}{Functionally Graded Material}
\newacronym{edx}{EDX}{Energy Dispersive X-ray}
\newacronym{psi2}{PSI-2}{Plasma Surface Interaction - 2}
\newacronym{cul}{Cu L$_{\alpha 1,2}$}{Copper L alpha$_{1,2}$ emission line}
\newacronym{edm}{EDM}{Electrical Discharge Machining}
\newacronym{se}{SE}{Secondary Electron}
\newacronym{bse}{BSE}{Backscattered Electron}
\newacronym{il}{IL}{In Lens Electron}
\newacronym{z}{Z}{Atomic number}
\newacronym{wilma}{WILMA}{Wolfram Infiltrations Maschine}
\newacronym{ccz}{\ch{CuCrZr}}{copper chromium zirconium alloy}
\newacronym{w}{\ch{W}}{tungsten}
\newacronym{wfw}{W$_f$/W}{tungsten fibre reinforced tungsten}
\newacronym{wf}{W$_f$}{tungsten fibre}
\newacronym{iter}{ITER}{International Thermonuclear Experimental Reactor}
\newacronym{elm}{ELM}{Edge-Localised Mode}
\newacronym{naoh}{\ch{NaOH}}{sodium hydroxide}
\newacronym{naf}{\ch{NaF}}{sodium fluoride}
\newacronym{hf}{\ch{HF}}{hydrogen fluoride}
\newacronym{h}{\ch{H2}}{hydrogen}
\newacronym{lrp}{LRP}{liquid ring pump}
\newacronym{wf6}{\ch{WF6}}{tungsten hexafluoride}
\newacronym{ar}{\ch{Ar}}{Argon}
\newacronym{cvdw}{CVD-W}{\acl{cvd} coating with \acl{w}}
% --- Zusätzlich eigene Glossare für Symbole und Indizes ---


% Symbole
\newglossaryentry{theta}{type=symbols,
  name={\ensuremath{\Theta}},
  description={Contact angle between droplet and surface},
  sort=theta,
  symbol={\ensuremath{^\circ}}}

\newglossaryentry{Ea}{type=symbols,
  name={\ensuremath{E_\text{A}}},
  description={Activation energy},
  sort=Ea,
  symbol={J}}

% Indizes
\newglossaryentry{max}{type=indices,
  name={max},
  description={Maximum value}}

\newglossaryentry{f}{type=indices,
  name={f},
  description={Fluorescence}}


    % Verzeichnisse
    % Abkürzungsverzeichnis ins TOC

    \printglossary[type=\acronymtype, title=List of Abbreviations, toctitle=List of Abbreviations]
    \printglossary[type=symbols, title=List of Symbols, toctitle=List of Symbols]
    \printglossary[type=indices, title=List of Indices, toctitle=List of Indices]

    \printbibliography[heading=bibintoc]
    \listoffigures
    \listoftables
    \appendix
    \chapter{Anhang}
Hier steht dein Anhang...
    
    
\end{document}
