%authors: Arne Lüken and Lukas Griesberg 02/2022
%please send comments and additions to Arne Lüken arne.lueken@avt.rwth-aachen.de
\chapter{Wissenschaftliches Schreiben}

\section{Schreibstil}
    Ein wissenschaftlicher Text sollte in der Lage sein, komplexe Themen verständlich zu vermitteln. Sätze und Formulierungen sind hierfür möglichst kurz und klar zu gestalten. Vermeiden Sie unnötige Informationen und Wörter. Alle Wörter, die ausgelassen werden können, ohne den Sinn zu verändern, sollten ausgelassen werden. Der Sprachstil ist sachlich zu halten; ein Spannungsaufbau sollte nicht stattfinden. Verwenden Sie nach Möglichkeit Aktiv anstatt Passiv. Speziell im Englischen macht es den Text lebendiger:
    \begin{compactitem}
        \item \underline{Passiv:} "Der Katalysator ist auf der Nickelelektrode aufgebracht."
        \item \underline{Aktiv:} "Die Elektrode besteht aus einer mit Katalysator beschichteten Nickelplatte."
    \end{compactitem}
    Verwenden Sie das Präsens für allgemeine Informationen, Bezug auf Graphen oder Abbildungen, sowie bei der Interpretation von Ergebnissen und das Präteritum für das Zitieren von Forschungsergebnissen und bei Bezug auf eigene Messungen und Ergebnisse (z.B. "...wurde gemessen."). Achten Sie zudem auf eine einheitliche Wortwahl. Sprechen Sie beispielsweise nicht abwechselnd von "Mikrogelen", "Hydrogelen" oder "Partikeln", sondern beschränken Sie sich auf einen dieser Begriff.

\section{Struktur von Text und Graphen}

    Bevor mit dem Schreiben begonnen wird, sollte ein roter Faden erstellt werden. Zunächst sollten die Kapitel und Unterkapitel grob festgelegt werden, um einen Überblick über die Struktur der Arbeit zu erhalten. Falls Sie Unterkapitel verwenden, müssen es jeweils mindestens zwei Unterkapitel sein (z.B. 1.3.1. und 1.3.2.). Meist bietet es sich an, vor mehreren Unterkapiteln eine kurze Einleitung des Oberkapitels zu setzen. In der Regel sollten Unter-Unter-Kapitel (1.3.1.) ausreichend sein, um die Übersichtlichkeit der Arbeit zu gewährleisten.
    
    Bevor Sie zu schreiben anfangen, bietet es sich an, die Abbildungen und Graphen, die sie verwenden wollen, einzufügen und als Ausgangspunkt für Ihren Text zu nehmen. Sollten diese noch nicht fertig sein, kann es helfen, die erwarteten Ergebnisse als Skizze/Handzeichnung einzufügen und am Ende durch die fertigen Abbildungen zu ersetzen. Wichtig ist, dass jede verwendete Abbildung im Text referenziert wird. Ergebnisse, die Sie für wichtig erachten, jedoch nicht direkt ansprechen oder interpretieren, gehören in den Appendix. Alle relevanten Informationen sollten im Graphen oder der Bildunterschrift, welche bei Papern über mehrere Zeilen gehen darf, enthalten sein. Der Leser sollte ohne den Fließtext zu lesen, erkennen, was er auf der Abbildung sieht und was diese ausmacht, wie etwa relevante Versuchsbedingungen.
    
    Bei der Erstellung von Abbildungen und Graphen ist es empfehlenswert, Farbe nur falls unbedingt nötig zu verwenden. Es sollten keine Informationen verloren gehen, falls die Arbeit in schwarz-weiß gedruckt wird. Abbildungen sollten möglichst kontrastreich gestaltet werden. Falls farbige Abbildungen unvermeidbar sind, sollten Rot und Grün möglichst vermieden werden, da diese für farbenblinde Menschen problematisch sein können. Für Beschriftungen innerhalb von Abbildungen verwenden Sie bitte kurvige Linien (siehe Abbildung~\ref{fig:Beispielgraph}). Graphen sollten mit der Software \href{https://www.originlab.com/}{Origin} erstellt werden. Für Origin gibt es Vorlagen und eine \href{https://cvtwiki.avt.rwth-aachen.de/wiki/doku.php?id=sop:data_management:origin}{\textbf{\textit{Origin-Anleitung}}} im CVTwiki. Bilder und Schemata sollten immer einen Maßstabsbalken enthalten.
    
    \begin{figure}[h]	% Requires \usepackage{graphicx}
    		\begin{center}
    		  \includegraphics[width=\textwidth]{4_images/Beispiel-Graph.png}
    		  \caption[Beispielgraph]{Ein Beispielgraph mit Gestaltungshinweisen, die in \href{https://cvtwiki.avt.rwth-aachen.de/wiki/doku.php?id=sop:data_management:origin}{Origin} einfach und direkt angepasst werden können. Das Origintutorial kann im CVT Wiki unter https://cvtwiki.avt.rwth-aachen.de/wiki/doku.php?id=sop:data\_management:origin gefunden werden. Achtung, dieser Link funktioniert nicht im Code Editor wegen \_.}\label{fig:Beispielgraph}
    		\end{center}
    \end{figure}
    
    Um eine logische, nachvollziehbare Textstruktur zu erhalten, bietet sich die Verwendung von Absätzen an. Hierbei sollte jeder Absatz jeweils eine Idee behandeln und diese ausführen. In der Regel hat ein Absatz eine Länge von 5~-~10 Zeilen, so dass 3~-~5 Absätze auf eine Seite passen. Jeder Absatz sollte daher grob wie folgt strukturiert sein:
    \begin{compactitem}
        \item \underline{Topic Sentence:} Einleitender Satz, der die Grundaussage des Absatzes darlegt. Er beinhaltet sowohl das Thema (z.B. Mikrogele) als auch die Leitidee (z.B. anisometrisches Schwellen).
        \item \underline{Supporting Sentences:} Erklärung oder Unterstützung der Hauptidee durch Details in mehreren Sätzen.
        \item \underline{Concluding Sentence:} Fasst die wichtigsten Punkte zusammen und signalisiert das Ende des Absatzes. Er ist ähnlich strukturiert wie der Topic Sentence, beinhaltet aber mehr Informationen.
    \end{compactitem}
    
    
\FloatBarrier
\section{Gliederung}
Wissenschaftliche Arbeiten gliedern sich in der Regel in folgende Abschnitte:
\begin{enumerate}
    \item[] \textbf{Abstract} \vspace{-4 mm}
    \begin{compactitem}
        \item Sehr kurze Zusammenfassung der Thesis; Überblick für potenzielle Leser
        \item Inhalt:
        \begin{compactitem}
            \item Warum untersucht die Arbeit das Thema? (Motivation)
            \item Wie wurde das Thema behandelt? (Methoden)
            \item Was sind die Ergebnisse? (Ein bis zwei der wichtigsten Erkenntnisse, die am Ende herauskommen)
            %ONE general sentence
            %ONE sentence about the problem
            %state the objective
            %state the methods
            %state most significant results
            %take-home message
        \end{compactitem}
        \item Struktur:
        \begin{compactitem}
            \item Maximal 200 Wörter.
            \item Gegenwart, unpersönlicher Stil (vermeiden Sie \textit{ich} oder \textit{wir}).
        \end{compactitem}
    \end{compactitem}
    \item \textbf{Introduction / Motivation} \vspace{-4 mm}
    \begin{compactitem}
        \item Inhalt:
        \begin{compactitem}
            \item Warum wird das Thema untersucht? Was ist der Hintergrund der Arbeit? (Keine langen Ausschweifungen; kurz und prägnant Motivation erklären)
            \item Wer sind die Interessensgruppen? (z.B. für welche Anwendungen ist es interessant? Gibt es Produkte, die mithilfe der Ergebnisse entwickelt werden könnten? etc.)
            \item Um welche Problemstellung geht es genau und wie wird diese angegangen? 
            \item Was ist die Stuktur der Arbeit?
        \end{compactitem}
        \item Struktur:
        \begin{compactitem}
            \item Ca. eine Seite, höchstens zwei.
            \item Keine Unterkapitel.
        \end{compactitem}
    \end{compactitem}
    \item \textbf{Theoretical Background and State of the Art} \vspace{-4 mm}
    \begin{compactitem}
        \item Liefert Grundlage für die spätere Diskussion und Interpretation der Ergebnisse
        \item Inhalt:
        \begin{compactitem}
            \item Was sind die Grundlagen und Hintergründe des behandelten Themas?
            \item Wie ist der bisherige Stand der Forschung?
            \item Was wurde im behandelten Thema bereits am Institut getan?
            \item Was sind Schnittmengen zu anderen Forschungsfeldern? Wie werden die Methoden sonst angewendet?
        \end{compactitem}
        \item Struktur:
        \begin{compactitem}
            \item Unterkapitel ergeben Sinn.
            \item Es sollte eine visuelle Darstellung des State-of-the-art erstellt werden (z.B. chronologisch wie in Abbildung \ref{fig:ChronologicStateOfTheArt}).
        \end{compactitem}
    \end{compactitem}
    \item \textbf{Concept / Hypothesis} (optional) \vspace{-4 mm}
    \begin{compactitem}
        \item Inhalt:
        \begin{compactitem}
            \item Im Gegensatz zur Einleitung sollen hier aufbauend auf der wissenschaftlichen Lücke des State of the Arts die der Arbeit zugrundeliegenden Hypothesen detailliert dargestellt werden. Darauf aufbauend werden dann die spezifischen angewandten Methoden und die erwarteten Ergebnisse aufgelistet. Was? Warum?
        \end{compactitem}
        \item Struktur:
        \begin{compactitem}
            \item Circa eine Seite.
        \end{compactitem}
    \end{compactitem}
    \item \textbf{Materials and Methods} \vspace{-4 mm}
    \begin{compactitem}
        \item Genaue Beschreibung des verwendeten Versuchsaufbaus und -durchführung, damit die Versuche \textit{reproduzierbar} sind.
        \item Inhalt:
        \begin{compactitem}
            \item Kapitel sollte enthalten:
                 Materialien (inkl. Hersteller), Verwendete Software,
                 Mengen / Konzentrationen,
                 Versuchsaufbau,
                 Versuchsvorbereitung,
                 Kalibrierung,
                 Versuchsprotokolle
            \item \underline{Der Versuch sollte auf Basis dieses Kapitels replizierbar sein}
            \item Gehen Sie auf eventuell vorgenommene Veränderungen am Versuchsaufbau ein, die stattgefunden haben, und nennen Sie Probleme, die aufgetreten sind.
            \item \underline{Beachte:} Keine Ergebnisse und keine Bewertung in diesem Kapitel.
        \end{compactitem}
        \item Struktur:
        \begin{compactitem}
            \item Unterkapitel sind möglich.
            \item Tabellen können hilfreich sein.
            \item Versuchsaufbau als beschriftetes Foto / Skizze / Verfahrensfließbild.
            \item Kurz und prägnant.
        \end{compactitem}
    \end{compactitem}
    \item \textbf{Results and Discussion} \vspace{-4 mm}
    \begin{compactitem}
        \item Inhalt:
        \begin{compactitem}
            \item Alle relevanten Ergebnisse sollten aufgeführt werden und in den Kontext der Arbeit gebracht werden. Nicht für das Ergebnis der Arbeit relevante Ergebnisse sollten im Anhang untergebracht werden.
            \item Verwendung von Graphen und Schemata.
            \item Ergebnisse werden zunächst beschrieben (Results) und anschließend im wissenschaftlichen Kontext (Bezug zu State-of-the-art) analysiert und interpretiert (Discussion).
        \end{compactitem}
        \item Struktur:
        \begin{compactitem}
            \item Die Ergebnisse sollen nicht chronologisch, sondern aufeinander aufbauend dargestellt werden, wie man auch eine Geschichte erzählen würde.
            \item Der rote Faden sollte anhand von Abbildungen (Graphen) erkennbar sein.
            \item Results und Discussion können als getrennte Kapitel behandelt oder zusammengefasst werden.
        \end{compactitem}
    \end{compactitem}
    \item \textbf{Conclusion and Outlook} \vspace{-4 mm}
    \begin{compactitem}
        \item Zusammenfassung der zentralen Ergebnisse für jemanden, der die Arbeit nicht komplett gelesen hat.
        \item Inhalt:
        \begin{compactitem}
            \item Ergebnisse werden erneut aufgeführt und die in "Results and Discussion" erreichten Schlüsse werden wiederholt. 
            \item Es werden keine neuen Diskussionen / Punkte erörtert.
            \item \underline{Outlook:} Mögliche Ansätze für weiterführende Forschung. Vorschläge und Ideen, wie es weitergehen kann.
        \end{compactitem}
        \item Struktur:
        \begin{compactitem}
            \item 1-2 Seiten; je maximal eine Seite für Conclusion und Outlook.
            \item Kann je nach Bedarf in einzelne Kapitel für Conclusion und Outlook aufgeteilt werden.
        \end{compactitem}
    \end{compactitem}
\end{enumerate}

\begin{figure}[h!bt]	% Requires \usepackage{graphicx}
		\begin{center}
		  \includegraphics[width=1\textwidth]{4_images/Beispiel_StateArt.png}
		  \caption{Beispielhafte Visualisierung des State-of-the-art im Bereich "Magnetresonanztomographie (MRI) in Membranfiltration" aus den Jahren 2002 - 2018.}\label{fig:ChronologicStateOfTheArt}
		\end{center}
\end{figure}
