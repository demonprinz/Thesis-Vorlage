%!TEX root = ../../main.tex
\tikzset{external/prefix=tikz/Erklärung/}
% Seite mit eidesstattlicher Erklärung
\begin{otherlanguage}{ngerman}


\newcommand{\underlinedTitle}[1]{%
  \begin{tikzpicture}[baseline=(textnode.base)]
    \node[inner sep=0pt, outer sep=0pt, anchor=base, text width=\textwidth] (textnode) {%
      \setlength{\tabcolsep}{0pt}%
      \begin{tabular}{@{}p{\textwidth}@{}}
        #1
      \end{tabular}%
    };
    % unterstreicht jede Zeile
    \foreach \y in {-12pt,-26pt,-40pt} { % 12pt pro Zeile anpassen
      \draw ([yshift=\y]textnode.north west) -- ([yshift=\y]textnode.north east);
    }
  \end{tikzpicture}%
}


% Erkärung nach Vorlage von AVT/PT
\begin{center} {\LARGE Eidesstattliche Erklärung} \end{center}


\noindent
\field[30ex]{\LastName, \FirstName} \hspace{15ex} \field[30ex]{\Matriculation}\\
\phantom{}\hspace{6ex} Name, Vorname \hspace{32.5ex} Matrikelnummer

Ich versichere hiermit an Eides statt, dass ich die vorliegende \texttypeDE mit dem Titel: \vspace{0.2cm}\\
\underlinedTitle{\titleEN}\\
selbstständig und ohne unzulässige fremde Hilfe (insbes. akademisches Ghostwriting) erbracht habe.
Ich habe keine anderen als die angegebenen Quellen und Hilfsmittel benutzt; dies umfasst
insbesondere auch Software und Dienste zur Sprach-, Text- und Medienproduktion. Ich erkläre, dass
für den Fall, dass die Arbeit in unterschiedlichen Formen eingereicht wird (z.B. elektronisch, gedruckt,
geplottet, auf einem Datenträger) alle eingereichten Versionen vollständig übereinstimmen. Die Arbeit
hat in gleicher oder ähnlicher Form noch keiner Prüfungsbehörde vorgelegen.\vspace{-0.5cm}\\

\noindent\field[30ex]{\Place, \Date} \hspace{15ex} \field[30ex]{\Signature}\\
\phantom{}\hspace{8ex} Ort, Datum  \hspace{35.5ex} Unterschrift \vspace{0.1cm}\\
\textbf{Belehrung:}\\
\textbf{\S 156 StGB: Falsche Versicherung an Eides statt}

\noindent Wer vor einer zur Abnahme einer Versicherung an Eides statt zuständigen Behörde eine solche Versicherung
falsch abgibt oder unter Berufung auf eine solche Versicherung falsch aussagt, wird mit Freiheitsstrafe bis zu drei
Jahren oder mit Geldstrafe bestraft.

\textbf{\S 161 StGB: Fahrlässiger Falscheid; fahrlässige falsche Versicherung an Eides statt}

\noindent(1) Wenn eine der in den \S\S 154 bis 156 bezeichneten Handlungen aus Fahrlässigkeit begangen worden ist, so tritt Freiheitsstrafe bis zu einem Jahr oder Geldstrafe ein.

\noindent (2) Straflosigkeit tritt ein, wenn der Täter die falsche Angabe rechtzeitig berichtigt. Die Vorschriften des § 158 Abs. 2 und 3 gelten entsprechend.

\noindent Die vorstehende Belehrung habe ich zur Kenntnis genommen:\vspace{0.4cm}\\
\noindent\field[30ex]{\Place, \Date} \hspace{15ex} \field[30ex]{\Signature}\\
\phantom{}\hspace{8ex} Ort, Datum  \hspace{35.5ex} Unterschrift \vspace{0.1cm}\newpage
\end{otherlanguage}


\begin{center} {\LARGE Declaration} \end{center}

I hereby declare to have listed all aids used in this thesis alongside their application below. This concerns in particular software and services for language, text, and media production. 
I have been made aware by my supervisor of the proper and improper use of such aids in the thesis.\newline
In this thesis, I have
\begin{itemize}
	\item[$\Box$] In particular not used software and services for language, text, and media production. 
	\item[$\Box$] 
	used the following aids (Please also list how you applied each aid):\\\\


\centering
\begin{tabular}{ll} 
\toprule
Tool(s) used & Reason for use  \\
\midrule
DeepL        & Grammar/ Spell checker, Reformulating            \\
ChatGPT      & LaTeX \& Python debugging and syntax, finding sources \\
            
\bottomrule
\end{tabular}
%* Please fill out the table in Excel using the Abrreviations specified on this page, then import the table in https://www.latex-tables.com/
% Press Auto-Booktabs, then press generate text.
% Lastly, please add the command \midpointrule after the first line, e.g.,
% Section & Tool Used & Reasoning  \\\midrule 
\end{itemize}


 \vspace{2cm}
